\section{\mfname Simulator}
\label{sec:simulation}

We have developed a flexible, open source, discrete event simulator capable of simulating arbitrary \mfname models. The simulator is developed in Python3 and uses the Graphviz and Matplotlib libraries to visualize all outputs. The simulator is available for download at \url{https://github.com/SDC-UIUC/synthesis}.

The \mfname simulator takes as input plant and requirement files specified in the YAML format. The plant input file specifies all the cells with the operations they support and their times and all the conveyors with their lengths and the cells they connect to. The requirement input file specifies all the various requirements one would like to run against the plant. Each requirement contains a list of nodes with a single operation and a list of edges to link these nodes. Examples of both types of input files can be found at the link above. 

During a single execution of the simulator, it first parses the user input and then create graphs of the inputs in the DOT language. Each graph is then written to a PNG file to allow users to visually verify that the simulator constructed the plant/requirement graphs correctly. Next, the simulator executes the system for a specified amount of time using the baseline controller (others can be coded). To make the system run deterministically, the sources assign requirements in a round-robin fashion to widgets whenever they are instantiated. At every time step, the simulator moves widgets around in the plant, updates the states of each cell and logs various metrics: throughput, end-to-end delay, and the number of live widgets in the plant at a given time. At the end of execution, the simulator will output a log file to show the state of every cell in the system and all live widgets at every time step. Additionally, it creates and saves the plots for all the metrics listed above.